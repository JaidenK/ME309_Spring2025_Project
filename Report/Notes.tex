\documentclass[12pt,journal,compsoc]{IEEEtran}


%\usepackage[margin=1in]{geometry}
%\usepackage{enumitem}
%\usepackage{float}
%\usepackage{empheq}
%\usepackage{datetime}

\usepackage[shortlabels]{enumitem}
\usepackage{empheq}
\usepackage{url}
\usepackage{amsmath, amssymb}
\usepackage[export]{adjustbox} % also loads graphicx
\usepackage{textcomp}
\usepackage{color}
\usepackage{nicefrac}
\usepackage{cancel} % \cancel for strike through text but diagonal
\usepackage[normalem]{ulem} % for \sout strike through text

%\usepackage{steinmetz}
%\newcommand{\HW}{Hot Wheels\textsuperscript{\textregistered}}
\newcommand{\HW}{Hot Wheels}
%\newcommand{\Matchbox}{Matchbox\textsuperscript{\textregistered}}
\newcommand{\Matchbox}{Matchbox}

\newcommand{\revise}[1]{{\color{red}\textit{#1}}}
\newcommand*{\multicitedelim}{:}

\newcommand{\degree}{$^{\circ}$}
\newcommand{\degre}{^{\circ}}

\newcommand{\Ohm}{$\Omega$}

\newcommand{\Fbox}[1]{\fbox{$\displaystyle #1$}}

\newcommand{\ezfig}[2]{
\begin{figure}[t]
\centering
\includegraphics[width=\linewidth]{images/#1.png}
\caption{\label{fig:#1} #2}
\end{figure}
}

\newcommand{\ezfigstar}[2]{
\begin{figure*}[t]
\centering
\includegraphics[width=\linewidth]{images/#1.png}
\caption{\label{fig:#1} #2}
\end{figure*}
}

\newcommand{\code}[1]{{\fontfamily{pcr}\selectfont #1}}

\newcommand{\TODO}{{\color{red} TODO}}


\begin{document}

This is a notes document. If I have report-worthy ramblings, I'll try to toss them here. -Jaiden



\section{Introduction}
Nothing is obvious and all models are wrong. 

Our goal is to develop a model 

\ezfig{BallMidFlight}{The ball-chute system on terminal descent. }

\section{Zeroth Model}
Consider a point mass in a state of freefall. 


\section{What force of drag?}
\href{https://youtu.be/huSIx2DvVIc}{https://youtu.be/huSIx2DvVIc}

There are two equations for the force of drag,
\begin{align*}
F_D &= \frac12C_D\rho Av^2 \\
\vec{F_D} &= -b\vec{v}
\end{align*}
where $b=\frac{F_D}{v}$ is a proportionality constant showing the ratio between the drag force and the velocity. How to know which we should use?

Linear drag is valid for "small objects traveling at low speeds." But does our system count? We would have to test experimentally. 

How many significant figures do we have in our numbers?

\section{Appendix}

\section{Presentation/Report Appendix Ideas}
\begin{itemize}
\item Drogue chute design choice. Why cone? Why hole cut out of tip of cone? Why not the common designs I see when I google parachute? Well this design {\it is} common for storm drogues for watercraft. Why is it common there? Perhaps because 
\begin{itemize}
\item This simple shape is easy to construct out of sturdy robust materials, which is important for emergency equipment such as a storm drogue
\item Other drogue chute/general parachute designs are more efficient with their drag/weight ratio, which is important for mass-constrained systems like aircraft
\end{itemize}
Cross chutes exist. I was unaware of this design at the time.

\item Derivations. We can just present our equations and provide the derivations in an appendix. Anticipated question during presentation: how did you come up with that equation? We can answer "we've included a derivation of it in the appendix. You can read that after the presentation or we can go over now if you would like." Even if the derivation ends up being wrong, it may allow us to move on with the presentation. If we end up being wrong, (perhaps the model is missing something we chose to leave out) we should keep track of it as we keep track of all the sources of error. 
\end{itemize}

\end{document}