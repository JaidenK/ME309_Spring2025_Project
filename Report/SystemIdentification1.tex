\section{System Identification}

Consider our system of a parachute attached to a spherical mass by a string being thrown through the air. How can we model this system? The answer may seem obvious depending on your physics background, but we claim that nothing is obvious.\footnote{Any problem worth studying will not have an obvious answer. If the answer seems obvious, question your assumptions.}

First it's important to identify what properties of the system we're actually interested in. Here we've decided we're interested in modeling the trajectory of the center of mass of the ball. Some things we aren't interested in are the motion of the parachute, the mass of the ball, the elasticity of the ball, the color of the ball, the temperature of the room, the location of the moon in relation to the Earth, what you had for breakfast, etc. 

Everything in the universe is connected\footnote{This is readily apparent by considering gravity or electromagnetism.} and where we choose to draw the line may seem arbitrary. Obviously we won't be modeling every atom of our system and all the forces they're experiencing, but there are still a handful of reasonable options:

\begin{enumerate}
\item Point Mass, No Drag\footnote{ADD FIGURE: Freebody diagram}

The ball-chute system could be modeled as a single point mass in a vacuum traveling on a parabolic trajectory. This would certainly miss 

\item Point Mass, Linear Drag\footnote{ADD FIGURE: Freebody diagram}



\end{enumerate}

