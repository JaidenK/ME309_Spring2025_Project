\documentclass[12pt,english]{article}
\usepackage[T1]{fontenc} % https://tex.stackexchange.com/questions/664/why-should-i-use-usepackaget1fontenc
\usepackage{babel} %https://tex.stackexchange.com/questions/27740/whats-the-benefit-of-loading-babel-when-writing-in-english

%\usepackage[margin=1in]{geometry}
%\usepackage{enumitem}
%\usepackage{float}
%\usepackage{empheq}
%\usepackage{datetime}

\usepackage[shortlabels]{enumitem}
\usepackage{empheq}
\usepackage{url}
\usepackage{amsmath, amssymb}
\usepackage[export]{adjustbox} % also loads graphicx
\usepackage{textcomp}
\usepackage{color}
\usepackage{nicefrac}
\usepackage{cancel} % \cancel for strike through text but diagonal
\usepackage[normalem]{ulem} % for \sout strike through text

%\usepackage{steinmetz}
%\newcommand{\HW}{Hot Wheels\textsuperscript{\textregistered}}
\newcommand{\HW}{Hot Wheels}
%\newcommand{\Matchbox}{Matchbox\textsuperscript{\textregistered}}
\newcommand{\Matchbox}{Matchbox}

\newcommand{\revise}[1]{{\color{red}\textit{#1}}}
\newcommand*{\multicitedelim}{:}

\newcommand{\degree}{$^{\circ}$}
\newcommand{\degre}{^{\circ}}

\newcommand{\Ohm}{$\Omega$}

\newcommand{\Fbox}[1]{\fbox{$\displaystyle #1$}}

\newcommand{\ezfig}[2]{
\begin{figure}[t]
\centering
\includegraphics[width=\linewidth]{images/#1.png}
\caption{\label{fig:#1} #2}
\end{figure}
}

\newcommand{\ezfigstar}[2]{
\begin{figure*}[t]
\centering
\includegraphics[width=\linewidth]{images/#1.png}
\caption{\label{fig:#1} #2}
\end{figure*}
}

\newcommand{\code}[1]{{\fontfamily{pcr}\selectfont #1}}

\newcommand{\TODO}{{\color{red} TODO}}

\author{
	King, Jaiden\\
	\texttt{jaiden.king.035@my.csun.edu}
	\and
	Greeff, Guy\\
	\texttt{guy.greeff.241@my.csun.edu}
	\and
	Rojas, Richard\\
	\texttt{richard.rojas.717@my.csun.edu}
	\and
	Schnider, Nathan\\
	\texttt{nathan.schnider.455@my.csun.edu}
	\and
	Thomas, Ryan\\
	\texttt{ryan.thomas.732@my.csun.edu}
}
\title{Trajectory Estimation of a Ballistic Projectile Experiencing Aerodynamic Drag\\\large{An Exercise in Modeling Our World}}

\begin{document}
\maketitle

\tableofcontents

% PLACEHOLDER IMAGE 
% This is just here to show how to include a graphic
\begin{figure}[t]
\centering
\includegraphics[width=0.5\linewidth]{images/BallMidFlight.png}
\caption{\label{fig:BallMidFlight} A photograph of our system. The system consists of a projectile traveling on an arc through the air after being thrown with some initial velocity. A drogue chute has been attached with the intention of linearly varying the aerodynamic drag, though it may have more dynamic impacts than intended.}
\end{figure}

% To reference the figure by number: Fig.~\ref{fig:BallMidFlight}

\section{Introduction}

The natural world is full of fascinating phenomena. Our ancestors have worked tirelessly to develop the language of mathematics to describe these phenomena. While exact analytic solutions to some of the more complicated problems often do not exist, or are too impractical to solve by hand, there are numerical techniques to approximate the solutions to these problems. However, the numerical techniques are still quite laborious to compute. Powerful institutions could employ teams of computers to labor for hours solving problems of science and engineering that we would find trivial today. 

We live in an exciting point in history where the profession of computer has been fully automated by machines. And not just automated for powerful institutions, but also made accessible to nearly all of humanity. The amount of compute accessible to anyone reading this paper is beyond the comprehension of our ancestors who originally developed the mathematics we shall employ. 

Modern hardware is capable of computing small models in what feels like an instant, and with these models we can foresee the future. As engineers, we must be comfortable wielding this immense power. 

In this paper we shall identify some physical phenomena, develop a mathematical model of it, and develop computer software to compute the model. 


\section{Data Collection}
	Our data collection procedure was rather simple as increasing the complexity for this section would introduce more potential sources of error. We first found a location with good lighting and a consistent background, the location that we chose was one of the racquetball courts on campus. 
	Next using a tripod we set up our camera making sure to have a clear view of the projectiles trajectory ensuring that that the projectile was always in frame. This was done so that tracker (I think this is the right software) could extract the data points modeling the projectiles path. Once this was completed multiple trials were attempted being as consistent as possible. 
	In order to incorporate variable drag multiple chute designs were implemented. This was completed so that we could test the viability of our codes ability to model drag and its effect on trajectory. 


\section{System Identification}

Consider our system of a parachute attached to a spherical mass by a string being thrown through the air. How can we model this system? The answer may seem obvious depending on your physics background, but we claim that nothing is obvious.\footnote{Any problem worth studying will not have an obvious answer. If the answer seems obvious, question your assumptions.}

First it's important to identify what properties of the system we're actually interested in. Here we've decided we're interested in modeling the trajectory of the center of mass of the ball. Some things we aren't interested in are the motion of the parachute, the mass of the ball, the elasticity of the ball, the color of the ball, the temperature of the room, the location of the moon in relation to the Earth, what you had for breakfast, etc. 

Everything in the universe is connected\footnote{This is readily apparent by considering gravity or electromagnetism.} and where we choose to draw the line may seem arbitrary. Obviously we won't be modeling every atom of our system and all the forces they're experiencing, and yet there are still a handful of reasonable options:

\begin{itemize}
\item Point Mass, No Drag

The ball-chute system could be modeled as a single point mass in a vacuum traveling on a parabolic trajectory. This would certainly lead to inaccuracies once a large parachute is used, but it may be a decent approximation in low-drag scenarios. \footnote{ADD FIGURE: Freebody diagram}

\item Point Mass, Linear Drag 

The simplest way to incorporate air resistance into our model is as a force proportional to the velocity of the ball.\footnote{ADD FIGURE: Freebody diagram}

\item Point Mass, Velocity-Squared Drag

At higher velocities, the effect of drag tends to depart from a simple linear relationship towards a velocity squared relationship

\item Multi-body System

The intention of the parachute was simply to increase the drag of the ball, but it's certainly more complicated than that in reality. A more accurate model of the system would represent the ball and parachute as separate bodies, potentially even modeling the off-center attachment point of the parachute. 

\end{itemize}

With all these options available to us, we rely on the following principle: keep it simple. We don't want to over-complicate things unless we know that we have to. Following this principle, we'll start by modeling our system as a point mass with no drag. 

\section{Computer Software Design}
In this section we will cover a portion of the software design. Please refer to the source code available on GitHub for more details. 

\href{https://github.com/JaidenK/ME309\_Spring2025\_Project}{https://github.com/JaidenK/ME309\_Spring2025\_Project}

\subsection{Functional Requirements}
By this point we have a set of test data ready to be processed and a mathematical model that we would like to fit to that data. Before we write the code, we must define the requirements of the software. 

\begin{itemize}

\item Data Input

The software shall read in timestamped position data from a comma-separated-values text file. 

\item Test Description Database 

A method for the software to store descriptions and initial model parameters for each test shall be provided. 

\item Model Fitting

The software shall generate a trajectory model fitted to a subset of the sampled data. The fit shall minimize the square of the residuals.

\item Impact Location Prediction

The software shall estimate the impact location using the model. 

\item Plot Generation

The software will need to generate plots of various system properties, including but not limited to the position, velocity, and acceleration of the projectile. 

\end{itemize}

\subsection{Design Requirements}

The design of the code should strive to satisfy the following requirements.

\begin{itemize}

\item Modularity and Maintainability

Code will be module so that it can be be unit tested in isolation and to reduce coupling between code components. Code will be written with maintainability in mind so that we can re-use our code for future projects. In particular, the implementation of the model should be easily swappable. 

\item Generic Processing 

Code will not be tailored for specific test cases. Other than changing options, such as which data set to plot and which model type to use, there will be no need to modify code to get it to run properly between different data sets.

\end{itemize}

\subsection{System Design}
\begin{figure}[t]
\centering
\includegraphics[width=0.9\linewidth]{images/SystemDiagram.png}
\caption{\label{fig:SystemDiagram} Software system diagram.}
\end{figure}

The system diagram shown in Fig.~\ref{fig:SystemDiagram} shows how the MATLAB scripts fit in to the system. We have several video clips, one per toss of the projectile, and each is processed by the user using Tracker. This provides us with multiple independent CSV files, which we called the Test Database. Next, we have the meta data describing each test, such as a plain text description and the initial model parameters. This is stored in a switch statement in the source code. Finally, all the configuration options are grouped into a single Script Options file. 

\begin{figure}[t]
\centering
\includegraphics[width=0.9\linewidth]{images/MainScriptDiagram.png}
\caption{\label{fig:MainScriptDiagram} Main Script Diagram.}
\end{figure}

All of the inputs are used by the top level executable MATLAB script \texttt{MainScript.m}. Fig.~\ref{fig:MainScriptDiagram} attempts to show the relationships between the main script and the others. 

\section{Algorithms and Patterns}
Here we will highlight a few of the algorithms and coding patterns used. Again, the reader is encouraged to refer to the source code available on GitHub.

\subsection{Model Factory}
We knew that the Model implementation would change (going from a parabolic model to a linear drag model) while the interface to the model would remain unchanged (still has the same parameters and the output is just the equations of motion). We also wanted to be able to easily change which model we were using without editing all the lines of code referring to the model. To address this, we took inspiration from the Factory Pattern. The \texttt{GenerateModel.m} file implements the model factory, which returns the appropriate version of the model based on the provided function parameters.

\subsection{Numeric Differentiation}
Previously we had stated that the parabolic model would be used as a baseline. However, we can be even more based. Without using any specific model, we can use numeric differentiation to estimate the velocity and acceleration of the projectile from the position data. That is, calculating the velocity from the simple change in sampled position over change in time.

\begin{align*}
\text{velocity} = \frac{\Delta \text{position}}{\Delta \text{time}}
\end{align*}

And likewise for acceleration. 

This technique is included as a "reference signal" because it is frequently the default go-to technique for computing velocity and acceleration in real-time settings. Many engineers will likely encounter data computed in this way, so it should be familiar. As you'll see in the analysis sections, this technique is highly sensitive to noise in the signal. One of the advantages of using the model of best fit is that we can "see through" the noise.

\subsection{Gradient Descent}
The method used to fit our model to the sampled data is a custom variant of Gradient Descent. The Wikipedia article\footnote{\href{https://en.wikipedia.org/wiki/Gradient\_descent}{https://en.wikipedia.org/wiki/Gradient\_descent}} goes into greater detail and we will just summarize it here. In general, Gradient Descent allows you to find the minimum value of a function by sampling the gradient (i.e. slope) at some point and taking a small step "downhill". This is an iterative process where you just keep taking small steps downhill. The function we are minimizing is the sum of squared errors, where the error function is the difference between the sampled position and the model predicted position at that sample time. 

In our system this is a 5-dimensional landscape that we're walking downhill in. The 5 dimensions (the model parameters) are the initial position's x and y coordinates, initial velocity x and y coordinates, and the drag coefficient. 

Our special tweak to the algorithm was to, at each iteration, find the value of the drag coefficient which minimized the error. The gradient of the error function was then used to update the remaining 4 parameters. The implementation of this tweaked version is available in \texttt{GradientDescent\_v2.m}.

The reason we went with Gradient Descent is because it's known to converge even for non-linear systems, and we wanted a technique that would be applicable to as wide a variety of model types as possible. It's definitely possible we could improve the algorithms by linearizing the error function. 

\section{Results Analysis}

%% Test 1

\begin{figure}[t]
\centering
\includegraphics[width=0.9\linewidth]{images/Analysis1_Test1_Fig5_NoDrag.png}
\caption{\label{fig:Analysis1_Test1_Fig5_NoDrag} Fitting the parabolic model to sampled position data for a ball with no parachute. Left: Sampled position (blue) with the fit trajectory (orange). Right: Velocity and acceleration computed using numeric differentiation (blue) and the model's velocity and acceleration (orange). For low-drag scenarios, the parabolic model appears adequate.}
\end{figure}

So how does the parabolic model perform? We started by running it on some data of just the plain ball with no parachute attached, thrown in a low arc to minimize drag effects. This should be the best-case scenario for the parabolic model. The data is shown in Fig.~\ref{fig:Analysis1_Test1_Fig5_NoDrag}. 

At first glance, this model is actually looking pretty decent. There aren't any huge problems with the trajectory plot at this scale. Looking at the numeric differentiation, it's clear that there's plenty of noise in the signal. The velocity plot has some noise, and the model's velocity forms a smooth line through the middle of it. The noise in the signal makes the numeric differentiation's acceleration data useless, but the model's acceleration is a constant 1g as expected. If this projectile had been the only designed we cared about, we would probably just call it done here and saved ourselves a lot of work by not going straight to the linear drag model. 

%% Test 4

\begin{figure}[t]
\centering
\includegraphics[width=0.9\linewidth]{images/Analysis1_Test4_Fig5_NoDrag.png}
\caption{\label{fig:Analysis1_Test4_Fig5_NoDrag} Fitting the parabolic model to sampled position data for a ball with a parachute. The parabolic model is unable to adequately capture the motion of this system.}
\end{figure}

\begin{figure}[t]
\centering
\includegraphics[width=0.9\linewidth]{images/Analysis1_Test4_Err_NoDrag.png}
\caption{\label{fig:Analysis1_Test4_Err_NoDrag} A closer look at the failure of the parabolic model. Red vectors show the error for each sampled position.}
\end{figure}

Next we tried a scenario with higher drag. Fig.~\ref{fig:Analysis1_Test4_Fig5_NoDrag} shows the results. 
\section{System Identification 2}
\section{Results Analysis 2}
\section{Conclusion}

Remind the reader how frickin amazing and wonderful the world is and how the time we live in is so exciting.

Remind them how mind blowing it is that we can see through the noise like in Fig.~\ref{fig:Analysis2_Test4_Fig5_LinearDrag} by leveraging our knowledge of the underlying system mechanics. 

\TODO

\section{MISC (Remove before submitting)}


\newpage
% Linear Drag Model
\begin{align*}
\begin{cases} 
\displaystyle \ddot{\bf r}(t) = \left[-\frac{1}{m}e^{-bt/m}(b\dot{\bf r}_{0y}+gm)\right]\hat{y} + \left[-\frac{b}{m}\dot{\bf r}_{0x}e^{-bt/m}\right]\hat{x} 
\\
\displaystyle \dot{\bf r}(t) = \left[\frac{1}{b}e^{-bt/m}(b\dot{\bf r}_{0y}+gm)-\frac{gm}{b}\right]\hat{y} + \left[\dot{\bf r}_{0x}e^{-bt/m}\right]\hat{x} 
\\
\displaystyle {\bf r}(t) = \left[\frac{m}{b}(\dot{\bf r}_{0y}+\frac{mg}{b})(1-e^{-bt/m})-\frac{mgt}{b}\right]\hat{y} + \left[\frac{m\dot{\bf r}_{0x}}{b}(1-e^{-bt/m})\right]\hat{x} + {\bf r}_0
\end{cases}
\end{align*}
where ${\bf r}_0$ is the initial position and $\dot{\bf r}_0$ is the initial velocity. We will set $m=1$ and $b$ will be fit by gradient descent.

\begin{align*}
\text{velocity} = \frac{\Delta \text{position}}{\Delta \text{time}}
\end{align*}

\end{document}
